\documentclass[11pt,a4,reqno]{amsart}
\topmargin=0truecm \oddsidemargin=1truecm \evensidemargin=1truecm
\textheight=22cm
\textwidth=15cm
\usepackage[utf8]{inputenc}
\usepackage{amsfonts}
\usepackage{amsmath}
\usepackage{amsthm}
\usepackage{amssymb}
\usepackage{bm}
\usepackage{bbm}
\usepackage{caption}
\usepackage{color}
\usepackage{dsfont}
\usepackage{enumitem}
\usepackage{geometry}
\usepackage{graphicx}
\usepackage{hyperref}
\usepackage{mathrsfs}
\usepackage{mathtools}
\usepackage{pdfsync}
\usepackage{pgfplots}
\usepackage{verbatim}


\captionsetup[figure]{font=small}
\def\W{\mathbb{W}}


\def\wl{\par \vspace{\baselineskip}}
\def\squarebox#1{\hbox to #1{\hfill\vbox to #1{\vfill}}}
\def\qed{\hspace*{\fill}
        \vbox{\hrule\hbox{\vrule\squarebox{.667em}\vrule}\hrule}}
\newenvironment{solution}{\begin{trivlist}\item[]{\bf Solution:}}
                      {\qed \end{trivlist}}


\numberwithin{equation}{section}
\newtheorem{theorem}{Theorem}[section]
\newtheorem{proposition}[theorem]{Proposition}
\newtheorem{lemma}[theorem]{Lemma}
\newtheorem{corollary}[theorem]{Corollary}
\theoremstyle{definition}
\newtheorem{definition}[theorem]{Definition}
\newtheorem{example}[theorem]{Example}
\theoremstyle{remark}
\newtheorem{remark}[theorem]{Remark}
\newtheorem{assumption}{Assumption}
\renewcommand\theassumption{A}

\newcommand{\bb}[1]{\mathbb{#1}}
\newcommand{\x}{\mathbf{x}}
\newcommand{\y}{\mathbf{y}}

\newcommand{\bbl}[1]{\mathbbl{#1}}
\newcommand{\scr}[1]{\mathscr{#1}}
\newcommand{\cl}[1]{\mathcal{#1}}
\newcommand{\dsf}[1]{\mathds{#1}}
\newcommand{\bdf}[1]{\mathbf{#1}}
\newcommand{\frk}[1]{\mathfrak{#1}}
\newcommand{\sps}[1]{\textsuperscript{#1}}
\newcommand{\E}{\bb{E}}
\newcommand{\R}{\bb{R}}
\newcommand{\C}{\bb{C}}
\newcommand{\Z}{\bb{Z}}
\newcommand{\Var}{\text{Var}}
\newcommand{\Sphr}{\bb{S}}
\newcommand{\ind}{\mathbbm{1}}
\newcommand{\per}{\text{per}}
\newcommand{\pGq}[3]{{}_{#2}{#1}_{#3}}
\newcommand{\pFq}[2]{\pGq{F}{#1}{#2}}
\newcommand{\dg}{\dagger}
\newcommand{\bs}{\backslash}
\newcommand{\weave}{\leftrightsquigarrow}
\newcommand{\cra}{\curvearrowright}
\newcommand{\twine}[1]{\overset{#1}{\curvearrowright}}


\DeclareMathOperator*{\argmax}{arg\,max}
\DeclareMathOperator*{\argmin}{arg\,min}
\let\det\relax % "Undefine" \det
\DeclareMathOperator{\det}{det}
\DeclareMathOperator{\sign}{sign}
\DeclareMathOperator{\sinc}{sinc}
\DeclareMathOperator{\besq}{BESQ}
\DeclareMathOperator{\dbesq}{dBESQ}
\DeclareMathOperator{\pois}{Pois}
\DeclareMathOperator{\ent}{Ent}
\DeclareMathOperator{\Ran}{Ran}
%\let\span\relax % "Undefine" \span
\DeclareMathOperator{\Span}{span}

\newcommand{\beq}{\begin{equation}}
\newcommand{\eeq}{\end{equation}}
\newcommand{\bnum}{\begin{enumerate}}
\newcommand{\enum}{\end{enumerate}}

\newcommand{\refthm}[1]{Theorem~\ref{#1}}
\newcommand{\refprop}[1]{Proposition~\ref{#1}}
\newcommand{\reflemma}[1]{Lemma~\ref{#1}}
\newcommand{\refremark}[1]{Remark~\ref{#1}}

\newcommand{\twin}[3]{#1 \overset{#3}{\curvearrowright} #2}








\title{\bf Notes: BD-simulation Algorithm}
\author{A. Chee, Y. Liu, T. Tian}

\begin{document}
\maketitle

%\tableofcontents
%\newpage

\section{Introduction \& Notation}

\subsection{Preliminaries}

	For a continuous-time Markov process $X = (X_t)_{t \geq 0}$ on a space $E$, its transition semigroup $P = (P_t)_{t \geq 0}$ is defined by $P_tf(x) = \E_x[f(X_t)]$ for $t \geq 0$, $x \in E$ and $f \in C_0(X)$. The process $X$ is called \textit{Feller} if $P$ is strongly continuous, i.e. $t \mapsto P_tf$ is continuous in the uniform topology for any $f \in C_0(X)$, in which case $P$ has a densely defined generator $(G, \cl{D}(G))$ such that 
		\beq Gf = \lim_{t \to 0} \frac{P_tf - f}{t}, \quad f \in \cl{D}(G). \eeq
		
	
\subsection{Processes on $\R$}

	\begin{enumerate}
	
		\item The \textit{squared Bessel} ($\besq_m$) process $X^{(B, m)}$ of index $m \geq 0$ is a diffusion process on $\R_+$ that is a solution to the SDE
		\beq dX_t = (1 + m) dt + \sqrt{2X_t} \, dW_t \eeq
		where $W$ is a standard Weiner process. 
		
		It's transition semigroup, which we denote $Q^{(m)}$, has generator $B^{(m)}$ that satisfies
		\beq B^{(m)}f(x) = x f''(x) + (1 + m) f'(x), \quad x \in \R_+, f \in C^2(\R_+). \eeq
		
		We will denote $Q^{(0)} = Q$ and $B^{(0)} = B$. The $\besq_m$ processes are $1$-self-similar:
		\beq \label{eqn:self_similar_besq} ( \alpha X_t^{(B, m)} \mid X_0^{(B, m)} = x) \overset{(d)}{=} (X_{\alpha t}^{(B, m)} \mid X_0^{(B, m)} = \alpha x), \quad \forall x \in \R_+, \alpha > 0 . \eeq
		
		The measure $\mu_m(dx) = \frac{x^m}{\Gamma(m + 1)} \, dx$ is invariant for $Q^{(m)}$ and $Q^{(m)}$ is self-adjoint in $L^2(\frk{m}_m)$. 	
		
		\item The \textit{Laguerre} process $X^{L, m}$ of index $m \geq $ is a diffusion process on $\R_+$ that is a solution to the SDE 
		\beq dX_t = (1 + m - X_t) dt + \sqrt{2X_t} \, dW_t \eeq
		where $W$ is a standard Weiner process. 
		
		It's transition semigroup, which we denote $K^{(m)}$, has generator $L^{(m)}$ that satisfies
		\beq L^{(m)}f(x) = x f''(x) + (1 + m - x) f'(x), \quad x \in \R_+, f \in C^2(\R_+). \eeq
		
		We will denote $K^{(0)} = K$ and $L^{(0)} = L$. The semigroup $K^{(m)}$ has stationary distribution $\epsilon_m(dx) = \frac{x^m}{\Gamma(m + 1)} e^{-x} \, dx$ ($\epsilon = \epsilon^{(0)}$) and $Q^{(m)}$ is self-adjoint in $L^2(\epsilon_m)$. The Laguerre processes are the stationary analog of the $\besq$ processes and satisfy 
		\beq \label{eqn:besq_lag} K^{(m)}_t = Q_{e^t - 1} d_{e^{-t}} = d_{e^{-t}} Q_{1 - e^{-t}} \quad t \geq 0, \text{on } L^2(\R_+) . \eeq
		
	\end{enumerate}
	
	
\subsection{Processes in $\Z_+$}

	\begin{enumerate}
	
		\item The \textit{discrete squared Bessel} ($\dbesq_m$) process $\bb{X}^{(B, m)}$ of index $m$ is a birth-death process on $\Z_+$ that has rates
		\beq \lambda_n = n + m + 1, \quad \mu_n = n , \quad n \in \Z_+. \eeq
		
		Let its transition semigroup and generator be denoted $\bb{Q}^{(m)}$ and $\bb{B}^{(m)}$, respectively. The measure $\frk{m}_m(n) = \frac{\Gamma(n + m + 1)}{n! \Gamma(m + 1)}$ is invariant for $\bb{Q}^{(m)}$ and $\bb{Q}^{(m)}$ is self-adjoint in $\ell^2(\frk{m}_m)$. 

	
	
	\end{enumerate}

	


\section{Simulation Algorithms and Formulae}

\subsection{Base BESQ case} 
	Let $\Lambda$ be the Poisson kernel operator 
	\beq \Lambda f(x) = \sum_{n \in \Z_+} f(n) \frac{x^n}{n!} e^{-x} . \eeq
	Then, $\Lambda$ extends to $\ell^2(\Z_+) \to L^2(\R_+)$ and for $t \geq 0$, 
	\beq Q_t\Lambda = \Lambda \bb{Q}_t \quad \text{on } \ell^2(\Z_+). \eeq
	That is, if $f \in \ell^2(\Z_+)$ and $F = \Lambda f \in L^2(\R_+)$, then for a.e. $x \in \R_+$
	\beq \E_x[F(X_t^{B})] = \E[f(\bb{X}_t^{(B)}) \mid X_0 \sim \pois(x)] . \eeq
	
\subsection{Base Laguerre case} 
	The Poisson kernel operator $\Lambda$ extends to $\ell^2(\frk{n}) \to L^2(\epsilon)$ and for $t \geq 0$, 
	\beq L_t\Lambda = \Lambda \bb{K}_t \quad \text{on } \ell^2(\frk{n}). \eeq
	That is, if $f \in \ell^2(\frk{n})$ and $F = \Lambda f$, then for a.e. $x \in \R_+$
	\beq \E_x[F(X_t^{L})] = \E[f(\bb{X}_t^{(L)}) \mid X_0 \sim \pois(x)] . \eeq



	
\section{Discussion}

	\begin{enumerate}
		\item We observe that it takes $O(e^t)$ jumps in the $\besq$ process to reach time $t$. Heuristically, this comes from considering the extreme scenario when $X_t^{(B)}$ trends upwards so $X_t^{(B)}(\tau_n) \sim n$ where $\tau_n$ is the $n$th jump time. Then, since the inter-jump times are on average $(X_t^{(B)}(\tau_n))^{-1} \sim n^{-1}$, we have $\tau_N \sim \sum_{n = 1}^N n^{-1} \approx \log N$. Hence, we need $N \sim e^t$ for $\tau_N \sim t$. 
		
		This exponential time complexity can be overcome with the self-similarity of $X^{(B)}$, see \eqref{eqn:self_similar_besq}, which can be written $d_{\alpha} Q_{\alpha t} = Q_{t} d_\alpha$, where $d_\alpha f(x) = f(\alpha x)$ is the dilation operator. This comes with the tradeoff that the simulation estimate variance is increased, but the variance increase should be polynomial in the dilation factor. 
		
		Alternatively, we propose to overcome the exponentially time complexity described above with simulating the $\besq$ process via the Laguerre process, using the relation \eqref{eqn:besq_lag}, in which case we can similar exponentially large times for the $\besq$ process in linear time. 
		
		\item .

	\end{enumerate}
	

\bibliographystyle{unsrt}
\bibliography{Determinantal_Semigroups.bib}


\end{document}

